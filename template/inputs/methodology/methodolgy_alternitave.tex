\section{Methodology}

In this section, we describe the methodological framework proposed by \cite[Reid et al.]{reid2025linear}, which we aim to reproduce.
The core contribution is a technique to incorporate structural inductive biases (Topological Masking) into efficient Linear Transformers without breaking their $\mathcal{O}(N)$ complexity.

% Experiment 4.1 and 4.3 — Time Complexity & Ablation Studies
\subsection{Theoretical Framework: Linear Topological Masking}

Standard "vanilla" attention computes an $N \times N$ matrix $A = \text{softmax}(QK^\top)$, having $\mathcal{O}(N^2)$\cite{vaswani2017attention}.
To incorporate graph structure $\mathcal{G}$, topological masking modulates this matrix element-wise with a mask $\mathbf{M}_{\alpha}(\mathcal{G})$:
$$A_{masked} = \mathbf{M}_{\alpha}(\mathcal{G}) \odot A$$
While effective, this operation requires materializing the full $N\times N$ matrix, which is incompatible with Linear Attention methods that
decompose attention as $\phi(Q)(\phi(K)^\top V)$ to achieve $\mathcal{O}(N)$ complexity, where $\phi$ is a feature map $\phi: \mathbb{R}^d \rightarrow \mathbb{R}^m$ with $m \ll N$. \\

The method proposed be Reid et al. resolves this conflict by parameterizing the mask $\mathbf{M}_{\alpha}(\mathcal{G})$ as a function of the weighted adjacency matrix $\mathbf{W}$, specifically a power series $\mathbf{M}_{\alpha}(\mathcal{G}) := \sum_{k=0}^{\infty} \alpha_k \mathbf{W}^k$. Instead of computing this explicitly, the authors approximate it implicitly via Graph Random Features\cite{reid2025linear}.

% Experiment 4.2 — VITS (Vision Image Topological Sampling): Vision–Image Tasks (Graph + Image Integration)
\subsection{Graph Random Features}

To maintain linear complexity, the dense mask $\mathbf{M}_{\alpha}$ is approximated using Graph Random Features. The method constructs sparse feature vectors $\hat{\phi}_{\mathcal{G}}(v_i)$ for each node $v_i$ such that their dot product provides an unbiased estimate of the topological kernel\cite{reid2025linear}:
$$\mathbb{E}[\hat{\phi}_{\mathcal{G}}(v_i)^\top \hat{\phi}_{\mathcal{G}}(v_j)] = \mathbf{M}_{\alpha ij}$$
These features are generated via importance sampling of random walks on the graph. By simulating $n$ random walks starting from each node and halting with probability $p_{halt}$, the algorithm constructs sparse vectors where non-zero entries correspond to visited nodes. This allows the topological mask to be applied implicitly in the feature space\cite{reid2025linear}:
$$\text{Att}_{LR, \hat{\mathbf{M}}}(Q, K, V, \mathcal{G}) := D^{-1} (\hat{\Phi}_{Q, \mathcal{G}} (\hat{\Phi}_{K, \mathcal{G}}^\top V)),$$
$$D := \hat{\Phi}_{Q, \mathcal{G}}(\hat{\Phi}_{K, \mathcal{G}}^\top 1_N)$$
This formulation preserves the associativity of matrix multiplication, ensuring the entire operation scales linearly with the number of tokens N.

% Experiment 4.4 — PCTS (Point Cloud Temporal State prediction): Predicting Particle Cloud Temporal States
\subsection{Convergence and Ablation Methodology}

A key theoretical claim of the original paper is that the quality of the estimated mask and therefore also the downstream performance depend on the variance of the GRF estimator. The authors propose that increasing the number of random walks $n$ reduces this variance, allowing the approximation $\hat{\mathbf{M}}$ to converge towards the exact mask $\mathbf{M}_{alpha}$\cite{reid2025linear}.
\\
To validate this property, the methodology includes an ablation framework (Appendix C.2) where the halting probability $p_{halt}$ is fixed, and the number of walker $n$ is varied logarithmically. This isolates the effect of estimator variance on model accuracy, testing the hypothesis that performance improves when increasing $n$ until plateauing at the performance of the exact mask\cite{reid2025linear}.


% ======================== previous content ========================
% ==================================================================

% \textit{You can change the name of this section as you see fit.}\\
% In this section you should give a description of the methodological aspects of your work, for instance how you modified an existing method to perform a particular task or to overcome a particular limitation. If your project is about reproducibility, here you should describe the method presented in the original paper.
