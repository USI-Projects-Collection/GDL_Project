\section{Implementation}

% Experiment 4.1 and 4.3 — Time Complexity & Ablation Studies
\subsection{Time Complexity \& Ablation Studies}

% Experiment 4.2 — VITS (Vision Image Topological Sampling): Vision–Image Tasks (Graph + Image Integration)
\subsection{VITS: Vision--Image Tasks (Graph + Image Integration)}

% Experiment 4.4 — PCTS (Point Cloud Temporal State prediction): Predicting Particle Cloud Temporal States
\subsection{PCTS: Point Cloud Temporal State prediction}





% ======================== previous content ========================
% ==================================================================

% This section should be structured as follows (from the Reproducibility challenge template):

% ---

% Briefly describe what you did and which resources did you use. E.g. Did you use author's code, did you re-implement parts of the pipeline, how much time did it take to produce the results, what hardware you were using and how long it took to train/evaluate. 

% \subsection{Datasets}
% Describe the datasets you used and how you obtained them. 

% \subsection{Hyperparameters}
% Describe how you set the hyperparameters and what was the source for their value (e.g. paper, code or your guess). 

% \subsection{Experimental setup}
% Explain how you ran your experiments, e.g. the CPU/GPU resources and provide the link to your code and notebooks.

% \subsection{Computational requirements}
% Provide information on computational requirements for each of your experiments. For example, the number of CPU/GPU hours and memory requirements. You'll need to think about this ahead of time, and write your code in a way that captures this information so you can later add it to this section. 